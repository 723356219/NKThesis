% !TeX root = ../main.tex
% -*- coding: utf-8 -*-

\chapter{The Tikz Package}


The {\scshape pdf}\ package, where ``{\scshape pdf}'' is supposed to mean ``portable
graphics format'' (or ``pretty, good, functional'' if you
prefer\dots), is a package for creating graphics in an ``inline''
manner. It defines a number of \TeX\ commands that draw
graphics. For example, the code \verb|\tikz \draw (0pt,0pt) -- (20pt,6pt);|
yields the line \tikz \draw (0pt,0pt) -- (20pt,6pt); and the code \verb|\tikz \fill[orange] (1ex,1ex) circle (1ex);| yields \tikz
\fill[orange] (1ex,1ex) circle (1ex);.

\begin{figure}[h]
    \centering
    % \usetikzlibrary{arrows, decorations.pathmorphing, backgrounds, positioning, fit, petri, automata}
 
\definecolor{yellow1}{rgb}{1,0.8,0.2} 
 

\begin{tikzpicture}[->,>=stealth,shorten >=1pt,auto,node distance=2.8cm,semithick]
  \tikzstyle{every state}=[fill=yellow1,draw=none,text=black]
 
  \node[state]         (S) at (-6, 0)              {$S$};
  \node[state]         (xin1) at (-2, 3)           {$X^1_{in}$};
  \node[state]         (xin2) at (-2, 1)        {$X^2_{in}$};
  \node[state]         (xin3) at (-2, -1)       {$X^3_{in}$};
  \node[state]         (xin4) at (-2, -3)           {$X^4_{in}$};
  \node[state]         (xout1) at (0, 3)          {$X^1_{out}$};
  \node[state]         (xout2) at (0, 1)        {$X^2_{out}$};
  \node[state]         (xout3) at (0, -1)   {$X^3_{out}$};
  \node[state]         (xout4) at (0, -3)           {$X^4_{out}$};
  \node[state]         (xin5)  at (3, -2)   {$X^5_{in}$};
  \node[state]         (xout5) at (5, -2)   {$X^5_{out}$};
  \node[state]         (DC) at (7, 2)           {$DC$};
 
  \path (S) edge[bend left=26]              node {$\infty$} (xin1)
            edge[bend left=12]              node {$\infty$} (xin2)
            edge[bend right=12]             node {$\infty$} (xin3)
            edge[bend right=26]             node {$\infty$} (xin4)
        (xin1) edge  node {$\alpha=1$} (xout1)
        (xin2) edge  node {$\alpha=1$} (xout2)
        (xin3) edge  node {$\alpha=1$} (xout3)
        (xin4) edge  node {$\alpha=1$} (xout4)
        (xin5) edge  node {$1$} (xout5);
  \draw[->] (xout1) to[out=-30,in=150] node {$\beta$} (xin5);
  \draw[->] (xout2.east) to[out=-15,in=165] node [below] {$\beta$} (xin5);
  \draw[->] (xout3.east) to[out=0,in=180] node [below] {$\beta$} (xin5.west);
  \draw[->] (xout1) to[out=-5,in=175] node {$\infty$} (DC);
  \draw[->] (xout5) to[out=40, in=-120] node {$\infty$} (DC);
\end{tikzpicture}
 
    \caption{\label{fig:exmaple1} 示例图1}
\end{figure}

In a sense, when you use {\scshape pdf}\ you ``program'' your graphics, just
as you ``program'' your document when you use \TeX.  You get all
the advantages of the ``\TeX-approach to typesetting'' for your
graphics: quick creation of simple graphics, precise positioning, the
use of macros, often superior typography. You also inherit all the
disadvantages: steep learning curve, no \textsc{wysiwyg}, small
changes require a long recompilation time, and the code does not
really ``show'' how things will look like.





\begin{figure}
    \centering
     %!TEX program = xelatex

 
\begin{tikzpicture}[node distance=2cm]
 %定义流程图具体形状
\node (start) [startstop] {Start};
\node (in1) [io, below of=start] {Input};
\node (pro1) [process, below of=in1] {Process 1};
\node (dec1) [decision, below of=pro1, yshift=-0.5cm] {Decision 1};
\node (pro2a) [process, below of=dec1, yshift=-0.5cm] {Process 2a};
\node (pro2b) [process, right of=dec1, xshift=2cm] {Process 2b};
\node (out1) [io, below of=pro2a] {Output};
\node (stop) [startstop, below of=out1] {Stop};
 
 %连接具体形状
\draw [arrow](start) -- (in1);
\draw [arrow](in1) -- (pro1);
\draw [arrow](pro1) -- (dec1);
\draw [arrow](dec1) -- (pro2a);
\draw [arrow](dec1) -- (pro2b);
\draw [arrow](dec1) -- node[anchor=east] {yes} (pro2a);
\draw [arrow](dec1) -- node[anchor=south] {no} (pro2b);
\draw [arrow](pro2b) |- (pro1);
\draw [arrow](pro2a) -- (out1);
\draw [arrow](out1) -- (stop);
\end{tikzpicture}

    \caption{\label{fig:exmaple2} 示例流程图2}
\end{figure}
